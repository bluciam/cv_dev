\newcommand{\siglabib}[1]{{\emph{\small\uppercase {#1}}}}
\newcommand{\sigla}[1]{{\small\uppercase {#1}}}
\newcommand{\jobt}[1]{\textbf{\small {#1}}}


\renewcommand{\labelitemi}{$\cdot$}

\newcommand{\Overview}{Aperçu}
\newcommand{\wordEmail}{couriel}
\newcommand{\wordTel}{tél}
\newcommand{\wordWeb}{toile}
\newcommand{\wordBlog}{blog}
%TITLES
\newcommand{\titlePositions}{Postes}
\newcommand{\titleInvitedVisits}{Visites}
\newcommand{\titleInvitedTalks}{Conférences invitées}
\newcommand{\titleEducation}{Formation}

\newcommand{\titleReferences}{Références}
\newcommand{\titleLanguages}{Langues}
\newcommand{\titleContributions}{Contributions à la communauté académique}
\newcommand{\titleOrganization}{Organisation de conférences et d'ateliers}
\newcommand{\titleSkills}{Compétences et connaissances}
\newcommand{\titleLogiciels}{Logiciels de recherche développés}
\newcommand{\titleInterests}{Intérêts}
\newcommand{\titleSummary}{Visites, publications et présentations}
\newcommand{\titleAwards}{Prix}

%Unis, cities, countries
\newcommand{\unsw}{University of New South Wales (\sigla{unsw}), \nameAustralia}
\newcommand{\uts}{University of Technology, (\sigla{uts}), \nameAustralia}
\newcommand{\nameLaval}{Université Laval}
\newcommand{\nameQuebec}{Québec}
\newcommand{\nameNeuchatel}{Université de Neuchâtel}
\newcommand{\nameSwitzerland}{Suisse}
\newcommand{\nameUIS}{Université Industrielle de Santander}
\newcommand{\nameAustralia}{Australie}
\newcommand{\nameIndia}{Inde}
\newcommand{\nameJapan}{Japon}
\newcommand{\nameMumbai}{Mumbai}
\newcommand{\nameFrance}{France}
\newcommand{\nameGermany}{Allemagne}
\newcommand{\nameColombia}{Colombie}
\newcommand{\nameMontreal}{Montréal}

%POSITIONS
\newcommand{\nameAssLec}{Chargée de cours (associée)}
\newcommand{\nameLec}{Chargée de cours}
\newcommand{\nameAcademic}{Chargée de cours}
\newcommand{\nameResearcher}{Chercheure}
\newcommand{\nameCasualLec}{Chargée de cours}
\newcommand{\nameVisiting}{Chercheure invitée}
\newcommand{\nameWriter}{Rédactrice téchnique}

%Descriptions
\newcommand{\cosupervisorPhD}{Codirectrice de deux thèses de doctorat}
\newcommand{\assesortheses}%
  {Correctrice et conseillère de quatre mémoires de 4ème année}
\newcommand{\supportSoftEng}%
%  {Support dans l'évaluation et le syllabus aux courses de génie
  {Support aux courses de génie logiciel:}
\newcommand{\PTLP}{Membre de l'équipe du projet TransLucid}
\newcommand{\EDocuments}{Conception et spécification de documents électroniques}

%Entitties
\newcommand{\nameAusGov}{Gouvernement australien}
\newcommand{\nameFacultyEng}{Faculté de génie}
\newcommand{\nameGovQuebec}{Gouvernement du Québec}
\newcommand{\nameSynergy}{programme {SYNERGY}}
\newcommand{\minEducation}{Ministère d'éducation}
\newcommand{\nameFeesEx}{exemption des frais de scolarité}
\newcommand{\nameFacultyForest}{Faculté de foresterie et de géomatique}
\newcommand{\nameSecondLang}{Français, langue seconde}
\newcommand{\wordGrant}{subvention}
\newcommand{\wordFunding}{Financement}
\newcommand{\fundacionUIS}{Fondation \sigla{uis}}

\newcommand{\nameSeminar}{Séminaire en recherche}
\newcommand{\nameSummer}{programme d'été}
\newcommand{\nameScholarship}{bourse complète}
\newcommand{\nameYearsCompleted}{Quatre de cinq années complétées}
\newcommand{\fourfive}{quatre de cinq ans}
\newcommand{\wordAnd}{et}
\newcommand{\wordTitle}{Titre}
\newcommand{\average}{moyenne supérieure à}

%Languages
\newcommand{\langMother}{langue maternelle}
%\newcommand{\langMother}{mother tongue}
\newcommand{\langFluent}{couramment}
\newcommand{\langConversational}{de base}
\newcommand{\langCours}{en voie de perfectionnement}
%\newcommand{\langConversational}{conversational}
\newcommand{\langBasics}{notions}
\newcommand{\langEnglish}{anglais}
\newcommand{\langFrench}{français}
\newcommand{\langSpanish}{espagnol}
\newcommand{\langItalian}{italien}
\newcommand{\langGerman}{allemand}
\newcommand{\langLatin}{latin}
\newcommand{\langJap}{japonais}


%Months
\newcommand{\jan}{janvier}
\newcommand{\feb}{février}
\newcommand{\mar}{mars}
\newcommand{\apr}{avril}
\newcommand{\may}{mai}
\newcommand{\jun}{juin}
\newcommand{\jul}{juillet}
\newcommand{\aug}{août}
\newcommand{\sep}{septembre}
\newcommand{\oct}{octobre}
\newcommand{\nov}{novembre}
\newcommand{\dec}{décembre}

%Publications
\newcommand{\titlePublications}{Publications}
\newcommand{\pubEditors}{Œuvres éditées}
\newcommand{\pubBooks}{Livres}
\newcommand{\pubTranslators}{Livres traduits}
\newcommand{\pubChapters}{Chapitres de livres}
\newcommand{\pubJournals}{Articles en revues}
%\newcommand{\pubConferences}{Articles en compte-rendus de conférences}
\newcommand{\pubConferences}{Compte-rendus}
\newcommand{\pubTheses}{Thèses}
\newcommand{\pubSupervised}{Thèses supervisées}
\newcommand{\pubCompilations}{Compilations}
\newcommand{\pubsurl}{Liste complète des publications:}

%Diplomas
%\newcommand{\supervisor}{Supervisor}
\newcommand{\phdInformatics}{Ph.D., Informatique}
\newcommand{\nameGeomatics}{Sciences géomatiques}
\newcommand{\nameRemoteSensing}{Télédétection}
\newcommand{\bScElectronic}{B.Sc., Sciences informatiques et électroniques}
\newcommand{\bEngSystems}{B.Eng., Génie de systèmes informatiques}

%Other
\newcommand{\teaching}{Enseignement}
\newcommand{\supervisionUNSW}{Direction d'études à l'UNSW}
\newcommand{\specialClass}{Conférence invitée}
\newcommand{\presentations}{Présentations}
\newcommand{\presentation}{Présentation}
\newcommand{\consultant}%
  {Conseillère à la révision du contenu et la présentation du cours}
\newcommand{\tutor}{Enseignante-tutrice}
\newcommand{\third}{\SEW\ de troisième année}
\newcommand{\CourseAdmin}{Responsable de cours}
\newcommand{\javaCourseAdmin}{Responsable de cours et enseignante-tutrice}
\newcommand{\sewCourseAdmin}%
  {Responsable de cours, conseillère du curriculum et évaluatrice}
\newcommand{\ssdiSupport}%
  {Soutien technique des devoirs. Enseignante et évaluatrice}
%\newcommand{\RStutor}{Tutrice}
\newcommand{\RStutor}{Enseignante-tutrice en \emph{Télédétection}}
\newcommand{\RS}{\emph{Télédétection}}
\newcommand{\researchAssistant}{Assistante de recherche}
\newcommand{\sysAdmin}{Gérante du système informatique}
\newcommand{\sysAdminA}{Installation, entretien et sécurité d'un réseau
   hétérogène \sigla{ibm AS}/400}
%\newcommand{\sysAdminB}{Sécurité du réseau}
\newcommand{\sysAdminC}{Formation des utilisateurs et soutien}
\newcommand{\sysAdminD}{Conception et actualisation de la base des données 
   des pièces cons\-truites sur place}
\newcommand{\assistantManager}{Assistante de gestionnaire}
\newcommand{\assistantManagerA}{Gestionnaire de soir et planificatrice
   d'horaire}
\newcommand{\assistantManagerB}%
  {Responsable de stock et de commande des marchandises}

\newcommand{\subsidized}{Conférence subventionnée, exemption des frais}
\newcommand{\fullPaid}{Frais pris en charge}
\newcommand{\inviteOnly}{Par invitation seulement}
\newcommand{\inviteDemo}{Démonstration de logiciel}
\newcommand{\invitePresentation}{Présentation}

\newcommand{\programreviewer}{Membre du comité de programme et examinatrice}
\newcommand{\specialIssue}{Examinatrice pour l'Édition spéciale}
\newcommand{\specialIssueJournal}{de la revue}
\newcommand{\programmember}{Membre du comité de programme}
%\newcommand{\localChair}{Directrice de l'organisation locale}
\newcommand{\localChair}{Organisation locale}
\newcommand{\with}{conjointement avec}
\newcommand{\sameplace}{colocalisé avec}

\newcommand{\anita}{%
  Première lauréate du prix \emph{Anita Borg}, School of
  Computer Science and Engineering, University of New South Wales,
  Sydney, Australie.\\
  Présenté aux étudiantes sachant combiner des résultats académiques
  exceptionnels et leadership dans le soutien des femmes dans l'environnement
  de technologie ainsi que leur rôle dans la société.
}

\newcommand{\sponsored}{Parrainée par}
\newcommand{\cooperation}{en coopération avec}
\newcommand{\et}{et}
\newcommand{\lncs}{%
  Compte-rendu publié par \sigla{lncs}, Springer-Verlag.
}
\newcommand{\webMistress}{Budget, organisation et arrangements locaux,
   ainsi que webmestre.}
\newcommand{\director}{Directrice}
\newcommand{\wic}{%
  Fondatrice et promotrice du programme en amenant différentes
  ressources provenant de l'université et d'ailleurs pour les faire 
  travailler ensemble.\\
  Organisation de «LinuxFest».
}
\newcommand{\girlsWorkshops}{%
  Organisatrice pionnière et chef d'équipe du \emph{Girls in Science and
  Technology Workshops} accueillis par \sigla{cse}%: quatre ateliers annuels
%  d'une journée, chacun avec 60 filles de \mbox{11--12} ans d'origine variée.
}

\newcommand{\toolsWord}{\textsl{\small Outils}}
\newcommand{\educationalSW}{logiciels de soutien à l'éducation}
\newcommand{\blackboard}{logiciel d'éducation Blackboard}
%\newcommand{\blackboard}{logiciel de soutien à l'éducation Blackboard}
%\newcommand{\moodle}{logiciel de soutien à l'éducation Moodle}
\newcommand{\moodle}{logiciel d'éducation Moodle}
\newcommand{\multimedia}{présentation multimédia sur Mac}
\newcommand{\internal}{logiciels d'éducation développés sur place}
\newcommand{\linuxtools}{outils Linux}
%\newcommand{\shell}{langage de script shell}
\newcommand{\shell}{shell}
\newcommand{\typesetting}{logiciels de mise en page}
\newcommand{\graphics}{editeurs graphiques}
\newcommand{\markers}{logiciels de correction}
\newcommand{\simulation}{logiciel de simulation}
\newcommand{\admin}{logiciel administratif du AS/400}
\newcommand{\spreadsheet}{programmation de feuille de calcul}

\newcommand{\apa}{Bourses du gouvernement australien et de la Faculté de
                  génie de l'UNSW}
\newcommand{\laval}{Bourses du \nameSynergy, \nameGovQuebec\ et de la
                    \nameFacultyForest, Université Laval}

\newcommand{\progLangs}{Langages de programmation (C, C++, Java,
haskell, JavaScript, basic, cobol, Fortran)\\
Langages de script (bash, sh, awk, sed, grep, Perl, ISE)}
\newcommand{\tools}{
Linux, outils Unix, Ubuntu, Open Source, Fedora\\
ArcGIS, Git, LinkedIn, WordPress, GTK\\
logiciels de typographie (la suite logicielle du \LaTeX, PStricks, metapost)\\
éditeurs graphiques (gimp, xfig, gnuplot), tableurs\\
éditeurs de texte (vi, gedit, LibreOffice, emacs, Word),
serveurs de courriel, {mutt}\\
logiciels de présentation (PowerPoint, beamer, LibreOffice)\\
logiciels de multimedia (audacity, Cinelerra, PiTiVi, VLC, imovies)\\
rsync, sauvegarde de fichiers, exécution automatiquement des scripts (cron),
  LANs \\
intéressée par OpenStreetMap, FirefoxOS, OpenERP
}
\newcommand{\TKDteaching}{Instructrice de Taekwondo, enfants et adultes}
\newcommand{\homeEd}{Éducation multilingue à la maison, primaire et secondaire}
\newcommand{\translator}{Traductrice}
\newcommand{\interests}{%
  Voyager, gadgets électroniques, Yoga Bikram, la bonne nourriture, cinéma}

%Courses Taught
\newcommand{\Cours}{Cours}
\newcommand{\IntroGIS}
      {\noindent Responsable du bloc \emph{Introduction aux systèmes
       d'information géographique et Arc\sigla{gis}} dans les cours du
       programme du génie civil}
\newcommand{\ManagementLand}{Aménagement environnemental du territoire}
\newcommand{\Sanitation}{Génie environnemental et sanitaire}
\newcommand{\IntroCivil}{Introduction au génie civil}
\newcommand{\hci}{Interaction homme-machine}
\newcommand{\infoSystems}{Introduction aux systèmes d'information}

\newcommand{\SEW}{Ateliers de génie logiciel}
\newcommand{\SSDI}{Conception et mise en œuvre de systèmes logiciels}
\newcommand{\computing}[1]{Computing #1}
\newcommand{\introComp}{Introduction à la programmation: Haskell}
\newcommand{\Java}{Infographie et Java}

\newcommand{\hciLecture}{%
  Responsable des blocs \emph{Collaboration, Quantification, et Entrée-sortie}.}
\newcommand{\hciAdmin}{Responsable du cours \emph{\hci}.}
\newcommand{\hciOther}{Spécification ou relecture des devoirs et des épreuves;
 évaluation.}

\newcommand{\supervisor}{Directeur de recherche}
\newcommand{\cosupervisor}{Co-directeur}
\newcommand{\honourList}{Liste d'honneur du doyen}

\newcommand{\superspreadsheet}{%
Le logiciel \texttt{S$^3$} fut créé comme un interface entre le programmeur
et les librairies et le noyau du environnement d'exécution de TransLucid.

TransLucid est un langage de programmation mettant en œuvre le paradigme de la
programmation Cartesiènne. Le projet fut conçu et développé à l'UNSW
(\texttt{\small http://translucid.web.cse.unsw.edu.au}) et je fais partie de
cette équipe de recherche.  Le \texttt{S$^3$} permet la navigation et la
modification du contexte de \emph{runtime}
($n$-uples des dimensions et des valeurs), et
l'évaluation des expressions incluant des dimensions.  Autrement dit,
l'interface facilite les requêtes de valeurs de dimensions et évalue les
demandes d'un système TransLucid.

Le \texttt{S$^3$} est developpé en \texttt{gtkmm} avec C++.
}

\newcommand{\acms}{%
Le \texttt{ACMS} est un serveur cartographique,
multilingue, multiusager, collaboratif, interactif avec interface web.

Ce travail, fait dans le cadre de mon doctorat,
s'est divisé en deux parties:
\begin{enumerate}
\item  D'une part, l'utilisasion (et apprentissage) d'un
grand nombre d'outils informatiques et cartographiques:

  \begin{itemize}
    \item
    programmation web, JavaScript, JSP et servlets Java;
    \item
    langages de script (Perl, shell), de programmation (Java, C) et de
    programmation intensionnelle (ISE, intense);
    \item
    serveurs web, Apache et Tomcat;
    \item
    outils de bases de données, SQL, PostgreSQL et MySQL;
    \item
    logiciels cartographiques tels que «Generic Mapping Tools» (GMT) et GRASS;
    \item
    logiciels de typographie, PostScript et la suite logicielle du \LaTeX.
  \end{itemize}

\item
D'autre part, la conception d'une nouvelle
approche pour l'intégration de tous ces outils et de différentes sources de
données. Le comportement de ce serveur se déroulait dans un contexte
multidimensionnel paramétrisant tous ces outils et sources de données. Le
partage partiel ou total de ce contexte entre utilisateurs créait directement
un environnement de collaboration.
\end{enumerate}
Quand Doug Engelbart, prix Turing 1997 et
inventeur de la souris et de l'écran bitmap, vit mon système, il dit que
c'était exactement ce genre de chose que lui et ses collègues envisageaient
faire en 1968, mais ils n'avaient pas un modèle simple le permettant.
}

\newcommand{\engagements}{Engagement académic}
