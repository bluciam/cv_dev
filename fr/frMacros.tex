\newcommand{\siglabib}[1]{{\emph{\small\uppercase {#1}}}}
\newcommand{\sigla}[1]{{\small\uppercase {#1}}}
\newcommand{\jobt}[1]{\textbf{\small {#1}}}

\renewcommand{\labelitemi}{$\cdot$}

\newcommand{\Overview}{Aperçu}
\newcommand{\wordEmail}{couriel}
\newcommand{\wordTel}{tél}
\newcommand{\wordWeb}{toile}
\newcommand{\wordBlog}{blogue}

%Languages
\newcommand{\langMother}{langue maternelle}
\newcommand{\langFluent}{couramment}
\newcommand{\langConversational}{de base}
\newcommand{\langCours}{en voie de perfectionnement}
\newcommand{\langBasics}{notions}
\newcommand{\langEnglish}{anglais}
\newcommand{\langFrench}{français}
\newcommand{\langSpanish}{espagnol}
\newcommand{\langItalian}{italien}
\newcommand{\langGerman}{allemand}
\newcommand{\langLatin}{latin}
\newcommand{\langJap}{japonais}

\newcommand{\langskill}{Parlé, ecrit}
\newcommand{\other}{Outre}
\newcommand{\germanlevel}{niveau B1, Goethe Institut}

%TITLES
\newcommand{\profile}{Profil professionnel}
\newcommand{\snapshot}{Aperçu des compétences en informatique}
\newcommand{\current}{En cours}
\newcommand{\titlePositions}{Postes}
\newcommand{\pastPositions}{Postes précédents}
\newcommand{\titleInvitedVisits}{Visites}
\newcommand{\titleInvitedTalks}{Conférences invitées}
\newcommand{\titleEducation}{Formation}

\newcommand{\titleLanguages}{Langues}
\newcommand{\titleOrganization}{Organisation de conférences et d'ateliers}
\newcommand{\titleSkills}{Compétences et connaissances}
\newcommand{\titleLogiciels}{Logiciels de recherche développés}
\newcommand{\titleInterests}{Intérêts}
\newcommand{\titleSummary}{Visites, publications et présentations}
\newcommand{\titleAwards}{Prix}

%Descriptions
\newcommand{\profiledesc}{

J'ai +10 d'expérience en \sigla{TI} et plus d'un an en développement Web.
Je voudrais utiliser mes compétences en conception, documentation,
formation, leadership et communication dans la réalisation des logiciels et
résolution des problèmes dans une équipe de développement.  Je suis
toujours en train d'apprendre et d'expérimenter avec des technologies de
pointe, de lire des blogues technologiques, de participer aux MeetUps, ou en
twitter.

\newcommand{\RoR}{\textsl{Ruby on Rails}}

}
\newcommand{\web}{Programmation web avec {\RoR}, méthodes de
testing, \sigla{css} et Bootstrap, \sigla{html}}
\newcommand{\open}{Outils de développement, Linux, langage de scripting}
\newcommand{\teachingknow}
  {Connaissances en C, C++, \sigla{JSP}, java, haskell, ArcGIS}

%Unis, cities, countries
\newcommand{\unsw}{University of New South Wales, % (\sigla{unsw}),
  \nameAustralia}
\newcommand{\uts}{University of Technology, Sydney, %(\sigla{uts}),
  \nameAustralia}
\newcommand{\nameUIS}{Université Industrielle de Santander}
\newcommand{\nameLaval}{Université Laval}
\newcommand{\nameQuebec}{Québec}
\newcommand{\nameNeuchatel}{Université de Neuchâtel}
\newcommand{\nameSwitzerland}{Suisse}
\newcommand{\nameAustralia}{Australie}
\newcommand{\nameIndia}{Inde}
\newcommand{\nameJapan}{Japon}
\newcommand{\nameFrance}{France}
\newcommand{\nameGermany}{Allemagne}
\newcommand{\nameColombia}{Colombie}
\newcommand{\nameMontreal}{Montréal}

%POSITIONS
\newcommand{\nameLec}{Chargé de cours}
\newcommand{\nameAcademic}{Chargé de cours}
\newcommand{\nameResearcher}{Chercheur}
\newcommand{\nameVisiting}{Chercheur invitée}
\newcommand{\facultymember}{Membre du corps professoral}
\newcommand{\assistant}{Assistant de laboratoire}
\newcommand{\supervisor}{Superviseur}
\newcommand{\consultant}{Consultant}
\newcommand{\webdev}{Développeur web}
\newcommand{\nameWriter}{Rédacteur technique}

%Descriptions
\newcommand{\webdevA}{De la création au déploiement du site web en \RoR\
d'un StartUp montréalais. Seule personne technique. En cours}
\newcommand{\webdevB}{Conception et développement front-end en utilisant
\sigla{CSS} et \sigla{HTML}}
\newcommand{\writerDesc} 
  {Transformation des entrevues en demandes de subvention}

%
\newcommand{\nameSecondLang}{Français, langue seconde}
\newcommand{\nameSeminar}{Séminaire en recherche}
\newcommand{\nameSummer}{programme d'été}
\newcommand{\nameScholarship}{bourse complète}
\newcommand{\nameYearsCompleted}{Quatre de cinq années complétées}
\newcommand{\fourfive}{quatre de cinq ans}
\newcommand{\wordAnd}{et}
\newcommand{\wordTitle}{Titre}

%Months
\newcommand{\jan}{janvier}
\newcommand{\feb}{février}
\newcommand{\mar}{mars}
\newcommand{\apr}{avril}
\newcommand{\may}{mai}
\newcommand{\jun}{juin}
\newcommand{\jul}{juillet}
\newcommand{\aug}{août}
\newcommand{\sep}{septembre}
\newcommand{\oct}{octobre}
\newcommand{\nov}{novembre}
\newcommand{\dec}{décembre}

%Publications
\newcommand{\titlePublications}{Publications}
\newcommand{\pubEditors}{Œuvres éditées}
\newcommand{\pubBooks}{Livres}
\newcommand{\pubTranslators}{Livres traduits}
\newcommand{\pubChapters}{Chapitres de livres}
\newcommand{\pubJournals}{Articles en revues}
%\newcommand{\pubConferences}{Articles en compte-rendus de conférences}
\newcommand{\pubConferences}{Compte-rendus}
\newcommand{\pubTheses}{Thèses}
\newcommand{\pubSupervised}{Thèses supervisées}
\newcommand{\pubCompilations}{Compilations}
\newcommand{\pubsurl}{Liste complète des publications:}

%Diplomas
\newcommand{\cs}{Informatique}
\newcommand{\phdInformatics}{Ph.D., Informatique}
\newcommand{\nameGeomatics}{Sciences géomatiques}
\newcommand{\nameRemoteSensing}{Télédétection}
\newcommand{\bScElectronic}{B.Sc., Sciences informatiques et électroniques}
\newcommand{\bEngSystems}{B.Eng., Génie de systèmes informatiques}


%Descriptions
\newcommand{\assesortheses}{correcteur et conseiller}
\newcommand{\curriculum}{réviseuse et éditrice de syllabus}
\newcommand{\specwriter}{rédactrice de spécifications}
\newcommand{\team}{membre d'équipe}
\newcommand{\docwriter}{rédacteur de documentation}
%\newcommand{\docwriter}{documentation}
%\newcommand{\tester}{évaluation de logiciel}
\newcommand{\tester}{évaluateur de logiciel}
\newcommand{\hw}{Conception, spécification, déploiement et correction de
travaux pratiques}
\newcommand{\softwarelab}{instructrice de laboratoire informatique}
\newcommand{\marker}{correctrice}
\newcommand{\sysAdmin}{Gérante du système informatique}
\newcommand{\tutor}{Enseignante-tutrice}
\newcommand{\researchAssistant}{Assistante de recherche}

%Courses Taught
\newcommand{\introComp}{Introduction à la programmation: Haskell}
\newcommand{\ManagementLand}{Aménagement environnemental du territoire}
\newcommand{\Sanitation}{Génie environnemental et sanitaire}
\newcommand{\IntroCivil}{Introduction au génie civil}

\newcommand{\hci}{interaction homme-machine}
\newcommand{\is}{systèmes d'information}
\newcommand{\gis}{systèmes d'information géographique}
\newcommand{\cg}{infographie}


%not used?
\newcommand{\CourseAdmin}{Responsable de cours}
\newcommand{\javaCourseAdmin}{Responsable de cours et enseignante-tutrice}
\newcommand{\sewCourseAdmin}%
  {Responsable de cours, conseillère du curriculum et évaluatrice}
\newcommand{\ssdiSupport}%
  {Soutien technique des devoirs. Enseignante et évaluatrice}
\newcommand{\sysAdminA}{Installation, entretien et sécurité d'un réseau
   hétérogène \sigla{ibm AS}/400}
%\newcommand{\sysAdminB}{Sécurité du réseau}
\newcommand{\sysAdminC}{Formation des utilisateurs et soutien}
\newcommand{\sysAdminD}{Conception et actualisation de la base des données 
   des pièces cons\-truites sur place}

%not used?
\newcommand{\programreviewer}{Membre du comité de programme et examinatrice}
\newcommand{\specialIssue}{Examinatrice pour l'Édition spéciale}
\newcommand{\specialIssueJournal}{de la revue}
\newcommand{\programmember}{Membre du comité de programme}


\newcommand{\localChairDesc}{directrice de conférence et de budget}
\newcommand{\localChair}{Organisation locale}

\newcommand{\director}{Directrice et Fondatrice}
\newcommand{\wic}{%
  Liaison entre les intervenants de la communauté, conception d'événements}
\newcommand{\girlsWorkshops}{%
  \emph{Girls in Science and Technology Workshops} (une initiative de Anita
  Borg)\\
  Organisatrice pionnière:
  coordonnateur,
  chef d'équipe:
  ateliers en \sigla{html}, Lego Mindstorms et iMovie,
  discours de bienvenue,
  analyse rétrospective.
}

\newcommand{\toolsWord}{\textsl{\small Outils}}
\newcommand{\educationalSW}{logiciels de soutien à l'éducation}
\newcommand{\blackboard}{logiciel d'éducation Blackboard}
%\newcommand{\blackboard}{logiciel de soutien à l'éducation Blackboard}
\newcommand{\moodle}{logiciel d'éducation Moodle}
%\newcommand{\moodle}{logiciel de soutien à l'éducation Moodle}
\newcommand{\multimedia}{présentation multimédia sur Mac}
\newcommand{\internal}{logiciels d'éducation développés sur place}
\newcommand{\linuxtools}{outils Linux}
%\newcommand{\shell}{langage de script shell}
\newcommand{\shell}{shell}
\newcommand{\typesetting}{logiciels de mise en page}
\newcommand{\graphics}{editeurs graphiques}
\newcommand{\markers}{logiciels de correction}
\newcommand{\simulation}{logiciel de simulation}
\newcommand{\admin}{logiciel administratif du AS/400}
\newcommand{\spreadsheet}{programmation de feuille de calcul}

% TOOLS
\newcommand{\progLangs}{Langages de programmation (C, C++, Java,
haskell, JavaScript, basic, cobol, Fortran)\\
Langages de script (bash, sh, awk, sed, grep, Perl, ISE)}
\newcommand{\tools}{
Linux, outils Unix, Ubuntu, Open Source, Fedora\\
ArcGIS, Git, LinkedIn, WordPress, GTK\\
logiciels de typographie (la suite logicielle du \LaTeX, PStricks, metapost)\\
éditeurs graphiques (gimp, xfig, gnuplot), tableurs\\
éditeurs de texte (vi, gedit, LibreOffice, emacs, Word),
serveurs de courriel, {mutt}\\
logiciels de présentation (PowerPoint, beamer, LibreOffice)\\
logiciels de multimedia (audacity, Cinelerra, PiTiVi, VLC, imovies)\\
rsync, sauvegarde de fichiers, exécution automatiquement des scripts (cron),
  LANs \\
intéressée par OpenStreetMap, FirefoxOS, OpenERP
}

\newcommand{\superspreadsheet}{%
Le logiciel \texttt{S$^3$} fut créé afin de démontrer l'utilisation des
documents électroniques contextuels, dans le cas d'étude du tableur
multidimensionnel. Il couvre la création, la conception, la recherche et
le développement du projet.} 

\newcommand{\acms}{%
Le projet \texttt{ACMS} fut développé comme preuve de concept. Les
itérations du développement devaient être flexibles et donner l'espace aux
changements continuels d'exigences, d'outils et de découvertes de recherche. 
Il a été un succès.
}

\newcommand{\translucid}{%
Dans l'équipe du projet, j'ai contribué à la conception et le planning
afin de créer le langage Translucid. J'ai conçu la structure de la
documentation adaptée à divers usagers. En étant le premier utilisateur, 
j'étais le testeur interactive principal.
}


